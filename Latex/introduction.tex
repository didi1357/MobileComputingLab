\section{Introduction}
\label{sec:intro}
\IEEEPARstart{T}{he} goal of this laboratory was to develop a mobile application which recognizes acitvities performed by the user of a smartphone. During the first part students were introduced to android as development platform as well as how sensors can be read on this platform. Further, a mobile application which recognizes a users activity based on the \gls{knn} algorithm had to be implemented. The decision which sensors to use, as well as which features to compute was also left open by the course organization. During the second part the \gls{knn} algorithm had to be compared to machine learning approaches. In detail a comparison to a completely pre-trained model, as well as a transfer-learning based model had to be done. This was done by extending the application from the first part of the laboratory.

Throughout the whole lecture it was tried to classify the same six categories of activity introduced by the \gls{wisdm} dataset\autocite{wisdm:dataset}:
\begin{itemize}
\item sitting
\item standing
\item walking
\item jogging
\item going downstairs
\item going upstairs
\end{itemize}